%-------------------------------------------------------------------------------
%	SECTION TITLE
%-------------------------------------------------------------------------------
\cvsection{Research Experience}


%-------------------------------------------------------------------------------
%	CONTENT
%-------------------------------------------------------------------------------


\begin{cventries}




%---------------------------------------------------------



  \cventry
    {Research Associate at \href{https://www.frc.ri.cmu.edu/}{Field Robotics Center}} % Job title
    {\href{https://www.ri.cmu.edu/}{Carnegie Mellon University}} % Organization
    {Pittsburgh, PA} % Location
    {Nov. 2017 - present} % Date(s)
    {
      \begin{cvitems} % Description(s) of tasks/responsibilities
      	\item \textbf{Developing Planning Software for DARPA Subterranean Challenge:} \begin{flushright} {\color{awesome} Adviser: Dr. Sebastian Scherer} \end{flushright}
		\begin{itemize}
			\item Developing custom DJI M-100 simulation with a rotary lidar in gazebo environment and implementing Global Trajectory Planner using OMPL libraries and Developing Local Trajectory Planner using custom trajectory libraries in ROS.
		\end{itemize}	
		\item \textbf{Developing Autonomy Software for Pipe Crawler robot:} \begin{flushright} {\color{awesome} Adviser: Prof. William (Red) Whittaker} \end{flushright}
		\begin{itemize}
			\item Developing 3D Perception software for pipe crawler robot using ToF Lidar and constructing 3D map of the environment using ICP.
			\item Developing online robot localization software using EKF and post processing localization using Factograph optimization (with GTSAM package).
		\end{itemize}
		\item \textbf{Developing Software/Hardware for Moon Rover Robot:} \begin{flushright} {\color{awesome} Adviser: Prof. William (Red) Whittaker} \end{flushright}
		\begin{itemize}
			\item Developing Hardware of a custom designed light weighted LiDar
			\item Developing Software for our custom built Lidar to construct a 3D point cloud
		\end{itemize}
      \end{cvitems}
    }





%---------------------------------------------------------

  \cventry
    {Research Assistant} % Job title
    {\href{http://www.uic.edu}{University of Illinois at Chicago}} % Organization
    {Chicago, IL} % Location
    {Jun. 2016 - Nov 2017} % Date(s)
    {
      \begin{cvitems} % Description(s) of tasks/responsibilities
		\item \textbf{Designing a fully integrated radar and communication system – named ComSens} \begin{flushright} {\color{awesome} Adviser: Dr. Sebastian Scherer} \end{flushright}
		\begin{itemize}
				\item Proposing the novel idea of integrating radar and communication systems using pilot symbols
				\item Formulating the optimization problem to design training signals
				\item Solving the optimization problem using Convex Optimization methods
				\item Published our results as a paper at Military Conference on Communication (IEEE MILCOM 2017)
		\end{itemize}
		\item \textbf{Optimizing Pilot Overhead for Ultra-Reliable Short-Packet Transmission} \begin{flushright} {\color{awesome} Adviser: Dr. Sebastian Scherer} \end{flushright}
		\begin{itemize}
				\item Channel estimation for short-packet communication
				\item Formulating  the optimization problem for finite-length packet transmission
				\item Analytically solving the optimization problem and finding the optimal training signal
				\item Published our results as a paper at International Conference on Communication (IEEE ICC 2017)
		\end{itemize}				      
      \end{cvitems}
    }





%---------------------------------------------------------
  \cventry
    {Research Assistant at Computer Vision Lab} % Job title
    {\href{http://www.sharif.ir/web/en}{Sharif University Of Technology}} % Organization
    {Tehran, Iran} % Location
    {Nov. 2014 - Feb. 2015} % Date(s)
    {
      \begin{cvitems} % Description(s) of tasks/responsibilities
			\item \textbf{Designing and Implementing an @Home robot for participating in AUT-CUP competitions} \begin{flushright} {\color{awesome} Adviser: Dr. Sebastian Scherer} \end{flushright}
			\begin{itemize}
					\item Designing the platform (Differential Derive Robot with 4DOF Lynx Robotic Arm on top)
					\item Controlling the robotic arm using inverse kinematic (analytical approach)
					\item Motion planning of the robot (using Artificial Potential Field)
					\item Our robot ranked 2nd in AUTCUP international Robotics competition (Artificial Intelligence League)
			\end{itemize}
      \end{cvitems}
    }

%---------------------------------------------------------
  \cventry
    {Research Assistant at Computer Vision Lab} % Job title
    {\href{http://aut.ac.ir/aut/}{Tehran Polytechnic University}} % Organization
    {Tehran, Iran} % Location
    {Jan. 2015 - Jul. 2015} % Date(s)
    {
      \begin{cvitems} % Description(s) of tasks/responsibilities
        \item \textbf{Completing and debugging an @Home Robot for ROBOCUP 2015 (Joao Pessoa, Brazil)}, Advisor: Dr. Shiri.\begin{flushright} {\color{awesome} Adviser: Dr. Sebastian Scherer} \end{flushright}
        \begin{itemize}
        \item Designing a digital circuit for gathering sensors data (Ultrasound, IR, Gyroscope, …) and IMUs over I2C BUS 
        \item Implementing Hough transform based algorithm for Robot Vision (using OpenCV)
        \item Our robot participated in ROBOCUP 2015 in Joao Pessoa, Brazil
        \end{itemize}
      \end{cvitems}
    }

%---------------------------------------------------------
%  \cventry
%    {Research Assistant at Telecommunications Lab} % Job title
%    {\href{http://en.sbu.ac.ir/sitepages/home.aspx}{Shahid Beheshti University}} % Organization
%    {Tehran, Iran} % Location
%    {Sep. 2013 - Dec. 2013} % Date(s)
%    {
%      \begin{cvitems} % Description(s) of tasks/responsibilities
%        \item Research on optimizing Power Line Communication Systems (PLC) with OFDM Modulation, Advisor: Prof. Afjei.
%        \begin{itemize}
%        		\item Simulating MMSE channel estimation method in PLC with OFDM Modulation(Using MATLAB)
%        		\item Simulating the same channel over different modulations (QAM, FSK, PSK) and comparing with OFDM
%        \end{itemize}
%      \end{cvitems}
%    }


\end{cventries}
